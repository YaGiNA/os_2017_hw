\documentclass[11pt,a4paper, uplatex]{jsarticle}
%
\usepackage{amsmath,amssymb}
\usepackage{bm}
\usepackage[dvipdfmx]{graphicx}
\usepackage{ascmac}
\usepackage{listings}
\usepackage{underscore}
\lstset{
    frame=single,
    numbers=left,
    tabsize=2
}
%
\setlength{\textwidth}{\fullwidth}
\setlength{\textheight}{40\baselineskip}
\addtolength{\textheight}{\topskip}
\setlength{\voffset}{-0.2in}
\setlength{\topmargin}{0pt}
\setlength{\headheight}{0pt}
\setlength{\headsep}{0pt}
%
\newcommand{\divergence}{\mathrm{div}\,}  %ダイバージェンス
\newcommand{\grad}{\mathrm{grad}\,}  %グラディエント
\newcommand{\rot}{\mathrm{rot}\,}  %ローテーション
%
\title{OS 第1回課題レポート}
\author{1510151  栁 裕太}
\date{\today}
\begin{document}
\maketitle
\section{課題1: 和訳}

\subsection{OSのインターフェイス}
オペレーティングシステム(以下、OS)の仕事は、コンピュータを複数のプログラムで共有したり、
ハードウェアハードウェア単体のサポートより便利なサービスを提供することである。
そのOSは低級ハードウェアを管理・取り出しを行い、それにより、例えば、
ワードプロセッサが自身のディスクハードウェアが使用中かどうか考慮する気にする必要がなくなる。
多重のハードウェアもまた、多くのプログラムにコンピュータを同時に共有及び実行
(あるいは実行と見せかける)することを許している。
最終的に、OSは制御された相互作用する方法を提供しており、
それによってこれらはデータを共有できたり、あるいは共に仕事をすることができるのである。

単一のOSはユーザにインターフェースを介してプログラム群を提供する。
よいインターフェースをデザインすることは難しいことがわかる。
一方に、我々はインターフェースをより簡単に正しく実行するために、シンプルで精密なものにしたがる。
もう一方に、我々はより洗練された特徴をアプリケーションに提供するよう誘惑されるかもしれない。
この緊張を解くトリックは、インターフェースをもっと普遍的に提供できるようにするために、
ほんの少しメカニズムに依拠するデザインにすることである。

この本は単一のOSを実態のある例として、
OSのコンセプトを説明するために利用する。
xv6というOSは、ケン・トンプソンとデニス・リッチーによる
UNIXOSの基本的なインファーフェースを提供し、
できるだけUNIX内部のデザインを模倣したものとなっている。
UNIXはメカニズムもよく内包した限定的なインターフェースで、驚くべき汎用性を提供する。
このインターフェースはにおいて成功してきており、現代のOS
―BSD, Linux, MacOS, Solaris, そして更に、限定的では在るが、Microsoft Windowsも―
はUNIXのようなインターフェースを所持している。
xv6を理解することは、これらのシステムやその他を理解するためのよいスタート地点となるのである。

図0-1に記載されたとおり、xv6は伝統的なkernelの形式をとっており、
プログラムを走らせるためのサービスのような特別なプログラムがある。
それぞれの走っているプログラム(プロセスと呼ぶ)は、
指示/データ/スタックを内蔵したメモリを保持している。
この指示はプログラムの計算が実装されている。
データは計算における変数である。
スタックは、プログラム処理のコールが構成されてある。

プロセスがカーネルサービスを呼び出す必要がある時、まずOSのインターフェース内にて手続きを呼び出す。
この手続きのことを"システムコール"と呼ぶ。
このシステムコールがカーネル内部に入り、カーネルがサービスと結果返しを行う。
それゆえプロセスはユーザスペースとカーネルスペースとの処理のやりとりを代行しているのである。
%
%
\end{document}
