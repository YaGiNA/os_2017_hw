\documentclass[11pt,a4paper, uplatex]{jsarticle}
%
\usepackage{amsmath,amssymb}
\usepackage{bm}
\usepackage[dvipdfmx]{graphicx}
\usepackage{ascmac}
\usepackage{listings}
\usepackage{underscore}
\lstset{
    frame=single,
    numbers=left,
    tabsize=2
}
%
\setlength{\textwidth}{\fullwidth}
\setlength{\textheight}{40\baselineskip}
\addtolength{\textheight}{\topskip}
\setlength{\voffset}{-0.2in}
\setlength{\topmargin}{0pt}
\setlength{\headheight}{0pt}
\setlength{\headsep}{0pt}
%
\newcommand{\divergence}{\mathrm{div}\,}  %ダイバージェンス
\newcommand{\grad}{\mathrm{grad}\,}  %グラディエント
\newcommand{\rot}{\mathrm{rot}\,}  %ローテーション
%
\title{オペレーティングシステム 第1回課題レポート}
\author{1510151  栁 裕太}
\date{\today}
\begin{document}
\maketitle
\section{課題1: 和訳}
\subsection{オペレーティングシステムのインターフェイス}
オペレーティングシステムの仕事は、コンピュータを複数のプログラムで共有したり、
ハードウェアハードウェア単体のサポートより便利なサービスを提供することである。
そのオペレーティングシステムは低級ハードウェアを管理・取り出しを行い、それにより、例えば、
ワードプロセッサが自身のディスクハードウェアが使用中かどうか考慮する気にする必要がなくなる。
多重のハードウェアもまた、多くのプログラムにコンピュータを同時に共有及び実行
(あるいは実行と見せかける)することを許している。
最終的に、オペレーティングシステムは制御された相互作用する方法を提供しており、
それによってこれらはデータを共有できたり、あるいは共に仕事をすることができるのである。

単一のオペレーティングシステムはユーザにインターフェースを介してプログラム群を提供する。
よいインターフェースをデザインすることは難しいことがわかる。
一方に、我々はインターフェースをより簡単に正しく実行するために、シンプルで精密なものにしたがる。
もう一方に、我々はより洗練された特徴をアプリケーションに提供するよう誘惑されるかもしれない。
この緊張を解くトリックは、インターフェースをもっと普遍的に提供できるようにするために、
ほんの少しメカニズムに依拠するデザインにすることである。
%
%
\end{document}
